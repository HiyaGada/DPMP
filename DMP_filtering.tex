\documentclass[12pt]{article} 
%
     % \usepackage{/Users/ravibanavar/PBMacBook_26_05_09/TEX_myhelp/macros}        
       \usepackage{LGobservers_macros,defs} 					
       \usepackage{epsfig} 
         \usepackage{graphicx}
         \usepackage{graphics}
         \usepackage{latexsym} 
         \usepackage{amsmath}
         \usepackage{psfrag}
         \usepackage{amsmath}
         \usepackage{amssymb}
         \usepackage{multicol}
         \usepackage{cite}
         \usepackage{url}
         \usepackage{bm} % bold math
         \usepackage{color} % change text color  
         \usepackage{tcolorbox}
%
          \def\deff{\stackrel{\triangle}{=}} 
%
%       Page dimensions
%
     %\renewcommand{\baselinestretch}{1.2} 
         \topmargin=-0.75in%-0.5in 
      %\headheight=0pt 
	%\headsep=0pt 
	\textheight=9.5in 
	%\textwidth=6.5in 
	%\oddsidemargin=0pt 
	%\pagestyle{plain} 
	%\footskip=0.65in 
% 

\title{Discrete Filtering on Lie Groups using the Maximum Principle 
\\ \small{Preliminary ideas}} 
% 
\author{ 
%
{Ravi N Banavar \thanks{Professor, Systems and Control
Engineering, IIT Bombay}  \\
(Collaborative work with Hiya Gada, Pieter van Goor, Robert Mahony )}
}
%
%
%\today
\date{}
\begin{document}
\maketitle 
% 
\thispagestyle{empty} 
%
% 
\textcolor{blue}{We formulate a candidate filtering problem in discrete time on the rotation group 
$\SOthree$ using an optimal control framework 
and propose a "Maximum Principle" approach to solve the problem.
\\
{\bf Aside}: I request your patience with my notation. For the past 15 years I am used to seeing the 
"hat" for skew-symmetrization; so I resort to the subscript $e$ to indicate an "estimate." We could change 
this later if it bothers the observer-community folks !}
%

 The governing kinematic equations of motion are
 %
 	\begin{align}
		R_{k+1} = R_k \exp (\hOmega{}_k T)
	\end{align}
 %
 where $\hOmega{}_k$ is the angular velocity at the $k$th instant and $T$ is the sampling interval.
A measurement of the direction of an inertial object $\eta \in \Stwo$, expressed in the body-frame of the robot 
 or satellite as
 %
 	\begin{align}
		y_k = R_k^T \eta
	\end{align}
 %
The noise processes are yet to be included in the picture; for now we treat the problem as that 
of an observer design. The observed state at stage $k$ is denoted by $R_{e_k}$ and the observer
is assumed to have a structure
%
 	\begin{align}
		R_{e_{k+1}} = R_{e_{k}} \exp (( \hOmega{} _k + \hM{}_k)T)
	\end{align}
 %%
where $\{ \hM_k : k = 0, \ldots, N-1\} $ is a sequence in $\sothree$ and $\hM{}_p$ is a 
function of $ \{ y_k : k = 0, \ldots, p\} $.

Define the error as $E_k \deff R_{e_k} R_{k}^T$ \textcolor{blue}{(spatial error ?)}. 
The error dynamics are then $
		E_{k+1} = [Ad_{R_{e_k}}   {\exp(\hM_k^T) } ] E_k
$
 %%
The optimal 
observer design problem would be:
 %
 	\begin{align}
		\min_{\hM_k: k=0, 1, \ldots, N - 1} \sum_{i=0}^{N} \trace{(E_i^T - I)}
		\nonumber \\
		\hbox{subject to $E_{k+1} = Ad_{R_{e_k}}   \exp(\hM_k^T) E_k$}
		\nonumber \\
		R_{e_{k+1}} = R_{e_{k}} \exp (( \hOmega{} _k + \hM{}_k)T)
		\nonumber \\
		y_0, y_1, \ldots, y_{N-1} \quad \hbox{given}
		\nonumber \\
		\hOmega{}_0, \hOmega{}_1, \ldots, \hOmega{}_{N-1} \quad \hbox{given}
	\end{align}
 %
\textcolor{blue}{Points to ponder: 
\been
%
	\item How do we bring in the dependance of $\hM_k$ on the measurements ?
	%
	Could we do it as follows ? 
	%
 	\begin{align}
		M_k = 	L_k (y_k - y_{e_k}) \quad L_k \in \R^{3 \times 3}
	\end{align}
 %%
	\item After addressing this problem, we could move on to the visual SLAM problem.
%
\een}
%
We follow \cite{phogat_DMP}.
%
\begin{assumption}
\label{ass:asm}
The following assumptions on the various maps are enforced throughout this article:
%
\begin{enumerate}[label=\textup{(A-\roman*)}, leftmargin=*, widest=b, align=left]
%
\item \label{asm:1} All maps are smooth.
%
\item \label{asm:2} There exists an open set \m{\mathcal{O} \subset \liea} such that: 
\begin{enumerate}[label=\textup{(\alph*)}]
\item the exponential map \m{\e:\mathcal{O} \rightarrow \e(\mathcal{O}) \subset \lieg} is a diffeomorphism, and 
\item the integration step \m{s_t} \clb{is in} \m{\e(\mathcal{O}) \text{\; for all\;} t;} see Figure \ref{fig:para}.
\end{enumerate}
\item \label{asm:3} The set of feasible control actions \m{\cset} 
(here, the innovation $M_t$) is convex for each \m{t=0,\ldots,N-1}.
%
\end{enumerate} 
\end{assumption}
 
   Assumption \ref{ass:asm} is crucial, as we shall see, in order to ensure the existence of multipliers that appear in the necessary optimality conditions for the optimal control problem. In particular, \ref{asm:1} ensures the existence of a convex approximation (known as a tent \cite{tent}) of the feasible region in a neighborhood of an optimal triple \m{(\op{R}_{e}_t,\op{E}_t,\op{M}_t)}. \ref{asm:2} gives the local representation of admissible trajectories \m{\left \{\left( R_{e_{t}},E_t \right)\right\}_{t=0}^{N} \in \mathcal{A}} in a Euclidean space. This assumption naturally holds in situations in which the discrete-time dynamics are derived from an underlying continuous-time system, thereby transforming our optimal control problem to a Euclidean space; first order necessary conditions for optimality are thereafter obtained using Boltyanskii's method of tents \cite{tent}. These first order necessary conditions are interpreted in terms of the (global) configuration space variables.  \ref{asm:3} leads to a pointwise non-positive condition of the gradient of the Hamiltonian over the set of feasible control actions, which is explained in detail in \secref{sec:proofDMP}.
%
\begin{equation}
\label{eq:sopt}
\begin{aligned}
\minimize_{\left\{M_t\right\}_{k=0}^{N-1}} &&& \mathscr{J} \left(R_e,E,,M\right) = \sum_{t=0}^{N-1} 
\trace \left(E_k - I \right) + \trace \left(E_N - I \right) \\
\text{subject to} &&&
\begin{cases}
\begin{cases} R_{e_{k+1}}^T R_{e_{k}} =  \exp( (\hat{\Omega}_k +  \hat{M}_k) T)  ,\\
E_{k+1} E_{k}^T = R_{e_{k}}  \exp(\hat{M}_k T) R_{e_{k}}^T , \\
 \end{cases} 
 \text{for each\;} k =0,\ldots,N-1, \\
 \text{under the above assumption \ref{ass:asm}.}
 \end{cases} 
\end{aligned} 
\end{equation}
%
%
\textcolor{red}{Once again, a note on the notation: 
\bei
%
\item To be consistent with our Automatica paper on the DMP, optimal quantities have a circle on top of them.
%
\item $G$ is the matrix Lie group, $\liea$ is the Lie algebra and $\liea^*$ is the dual.
%
\item The superscript on the Hamiltonian $\nu$ indicates the "abnormal" multiplier.
%
\item The states are $R_e$ (the estimate) and $E$ (the error in the estimate).
%
\item The adjoint variables are $\zeta$ (corresponding to the estimator dynamics) and $\xi$ 
(corresponding to the error dynamics).
%
\iten $M$ (the innovations) is the "control."
%
\ei
%
}
%
\textcolor{blue}{Our main result would be on the lines}
%
\begin{theorem}[{{\textcolor{blue}{Discrete-time PMP for a Lie Group Estimator}}}]
\label{thm:DMP}
Let \m{\{\op{M}_t\}_{t=0}^{N-1}} be an innovations process that solves the problem \eqref{eq:sopt} with \m{\{\left(\op{R_e}_t,\op{E}_t \right)\}_{t=0}^N} the corresponding state trajectory. Define the Hamiltonian 
\begin{align}\label{def:Hamiltonian}
& \hamdef \ni\hamvar \mapsto \nonumber \\
& \ham{\hamvar} \defas\nu \trace{\left(E_\tau - I \right)} + \left\langle\zeta,
\hat{\Omega}_\tau +  \hat{M}_\tau \right\rangle_{\liea} \nonumber\\
&\phantom{\ham{\hamvar} \defas} + \left\langle \xi,  R_{e_{\tau}} \hat{\Omega}_\tau 
R_{e_{\tau}}^T \rangle_{\liea}   \in \R, 
\end{align}
for \m{\nu \in \R.}
There exist 
\begin{itemize}[ leftmargin=*]
\item an adjoint trajectory \m{\left\{ \left(\zeta^t,\xi^t \right)\right\}_{t=0}^{N-1} \subset \liea^*\times \liea^*  }
and
\item a scalar \m{\nu \in \{-1, 0\} }
\end{itemize}
 such that, with \[\optraj{t}  \defas \left(t,\zeta^t,\xi^t,\op{R}_{e}_t,\op{E}_t, \op{M}_t \right),\] the following hold:
	\begin{enumerate}[leftmargin=*, label={\rm (MP-\roman*)}, widest=iii, align=left]
\item \label{main:dyn} state and adjoint system dynamics
\begin{align*}
state & \begin{cases}
\op{R}_{e_{t+1}}^T \op{R}_{e_{t}}  = {\mathcal{D}_{\zeta} \ham{\left(\optraj{t} \right)}},\\
\op{E}_{t+1} \op{E}_{t}^T = \mathcal{D}_{\xi} \ham{\left(\optraj{t} \right)}, 
\end{cases}\\
adjoint & \begin{cases}
\zeta^{t-1} = \\
%\ad{\e^{-\mathcal{D}_{\zeta} \ham{\left(\optraj{t} \right)}}}^* \zeta^{t} \\
%\phantom{\zeta^{t-1} }+\triv{\op{q}_t} \Big(\mathcal{D}_{q} \ham{\left(\optraj{t} \right)} \big)\\
%\phantom{\zeta^{t-1} }+\triv{\op{q}_t} \Big(\mu^t \mathcal{D}_{q} g_{t}\left(\op{q}_t,\op{x}_t \right) \Big),\\
\xi^{t-1} = 
%\mathcal{D}_{x} \ham{\left(\optraj{t} \right)} + \mu^t \mathcal{D}_{x} g_{t}\left(\op{q}_t,\op{x}_t \right),
\end{cases}
\end{align*}
\item \label{main:trans} transversality conditions
\begin{align*}
\zeta^{N-1}& = 
%\triv{\op{q}_N}\Big(\nu \mathcal{D}_{q} c_N \left(\op{q}_N,\op{x}_N \right) \Big) \\
%&+ \triv{\op{q}_N}\Big( \mu^N \mathcal{D}_{q} g_N \left(\op{q}_N,\op{x}_N \right)\Big),\\
%\xi^{N-1} = & \nu \mathcal{D}_{x} c_N \left(\op{q}_N,\op{x}_N \right) + \mu^N \mathcal{D}_{x} g_N \left(\op{q}%_N,\op{x}_N \right),
\end{align*}
\item \label{main:hmax} Hamiltonian non-positive gradient condition \label{it:hamnpg}
\[\ip{\mathcal{D}_{M} \ham{\left(\optraj{t}\right)}}{w-\op{M}_t} \leq 0 \quad \text{for all} \quad w \in \cset,\]
\item \label{main:ntriv} non-triviality condition \\
adjoint variables \m{\left\{\left(\zeta^t,\xi^t\right)\right\}_{t=0}^{N-1}}and the scalar \m{\nu} do not simultaneously vanish.
\end{enumerate}
\end{theorem}
\bibliography{ref} 
\bibliographystyle{siam} 
%
% 
\end{document} 
%%%%%%%%%%%%%%%%%%%%%%%%%%%%%%%
%
%
 %%%%%%%%%%%%%%%%%%%%%%%%%%%%%%%	
%
 \begin{equation}
\label{eq:sopt}
\begin{aligned}
\minimize_{\left\{u_t\right\}_{t=0}^{N-1}} &&& \mathscr{J} \left(q,x,u\right) = \sum_{t=0}^{N-1} c_t \left(q_t,x_t,u_t\right) + c_N \left(q_N,x_N\right)\\
\text{subject to} &&&
\begin{cases}
\begin{cases} q_{t+1} = q_t s_t \left(q_t,x_t\right),\\
 x_{t+1} = f_t \left(q_t,x_t,u_t\right), \\
 u_t \in \cset,
 \end{cases} 
 \text{for each\;} t =0,\ldots,N-1, \\
 g_t \left(q_t,x_t\right) \leq 0 \quad \text{for each\;} t =0,\ldots,N, \\
  b\left(q_0,x_0,q_N,x_N\right)=0,\\
 \text{under Assumption \ref{ass:asm}.}
 \end{cases} 
\end{aligned} 
\end{equation}
Our main result is the following:
\begin{theorem}[{{Discrete-time PMP}}]
\label{thm:DMP}
Let \m{\{\op{u}_t\}_{t=0}^{N-1}} be an optimal controller that solves the problem \eqref{eq:sopt} with \m{\{\left(\op{q}_t,\op{x}_t \right)\}_{t=0}^N} the corresponding state trajectory. Define the Hamiltonian 
\begin{align}\label{def:Hamiltonian}
& \hamdef \ni\hamvar \mapsto \nonumber \\
& \ham{\hamvar} \defas\nu c_\tau \left(q,x,u\right) + \left\langle\zeta,\e^{-1}\left(s_\tau \left(q,x\right)\right)\right\rangle_{\liea} \nonumber\\
&\phantom{\ham{\hamvar} \defas} + \left\langle \xi,f_\tau \left(q,x,u\right) \right\rangle  \in \R, 
\end{align}
for \m{\nu \in \R.}
There exist 
\begin{itemize}[ leftmargin=*]
\item an adjoint trajectory \m{\left\{ \left(\zeta^t,\xi^t \right)\right\}_{t=0}^{N-1} \subset \liea^*\times \left(\R^{\nox}\right)^*  }, covectors \m{ \mu^t \in \left(\R^{\nog{t}}\right)^*} for \m{t=1,\ldots, N,} and
\item a scalar \m{\nu \in \{-1, 0\} }
\end{itemize}
 such that, with \[\optraj{t}  \defas \left(t,\zeta^t,\xi^t,\op{q}_t,\op{x}_t, \op{u}_t \right),\] the following hold:
	\begin{enumerate}[leftmargin=*, label={\rm (MP-\roman*)}, widest=iii, align=left]
\item \label{main:dyn} state and adjoint system dynamics
\begin{align*}
state & \begin{cases}
\op{q}_{t+1} = \op{q}_t \e^{\mathcal{D}_{\zeta} \ham{\left(\optraj{t} \right)}},\\
\op{x}_{t+1} = \mathcal{D}_{\xi} \ham{\left(\optraj{t} \right)}, 
\end{cases}\\
adjoint & \begin{cases}
\zeta^{t-1} = \ad{\e^{-\mathcal{D}_{\zeta} \ham{\left(\optraj{t} \right)}}}^* \zeta^{t} \\
\phantom{\zeta^{t-1} }+\triv{\op{q}_t} \Big(\mathcal{D}_{q} \ham{\left(\optraj{t} \right)} \big)\\
\phantom{\zeta^{t-1} }+\triv{\op{q}_t} \Big(\mu^t \mathcal{D}_{q} g_{t}\left(\op{q}_t,\op{x}_t \right) \Big),\\
\xi^{t-1} = \mathcal{D}_{x} \ham{\left(\optraj{t} \right)} + \mu^t \mathcal{D}_{x} g_{t}\left(\op{q}_t,\op{x}_t \right),
\end{cases}
\end{align*}
\item \label{main:trans} transversality conditions
\begin{align*}
\zeta^{N-1}& = \triv{\op{q}_N}\Big(\nu \mathcal{D}_{q} c_N \left(\op{q}_N,\op{x}_N \right) \Big) \\
&+ \triv{\op{q}_N}\Big( \mu^N \mathcal{D}_{q} g_N \left(\op{q}_N,\op{x}_N \right)\Big),\\
\xi^{N-1} = & \nu \mathcal{D}_{x} c_N \left(\op{q}_N,\op{x}_N \right) + \mu^N \mathcal{D}_{x} g_N \left(\op{q}_N,\op{x}_N \right),
\end{align*}
\item \label{main:hmax} Hamiltonian non-positive gradient condition \label{it:hamnpg}
\[\ip{\mathcal{D}_{u} \ham{\left(\optraj{t}\right)}}{w-\op{u}_t} \leq 0 \quad \text{for all} \quad w \in \cset,\]
\item \label{main:comp} complementary slackness conditions
\begin{align*}
\mu^{t}_j g^j_{t}(\op{q}_t,\op{x}_t) = 0 \quad &\text{for all} \quad j=1,\ldots,\nog{t},\\
& \text{and} \quad t=1,\ldots,N, 
\end{align*}
\item \label{main:npos} non-positivity condition
\[ \mu^t \leq 0 \quad \text{for all}\quad t=1,\ldots,N,\]
\item \label{main:ntriv} non-triviality condition \\
adjoint variables \m{\left\{\left(\zeta^t,\xi^t\right)\right\}_{t=0}^{N-1}}, covectors \m{\left\{ \mu^t  \right\}_{t=1}^{N}}, and the scalar \m{\nu} do not simultaneously vanish.
\end{enumerate}
\end{theorem}
We present a proof of Theorem \ref{thm:DMP} in \secref{sec:proofDMP}. This discrete-time PMP on matrix Lie groups is a generalization of the standard discrete-time PMP on Euclidean spaces since the variable \m{q} in the combined state \m{\left(q,x\right)} evolves on the Lie group \(G\); consequently, the assertions of Theorem \ref{thm:DMP} appear different from the discrete-time PMP on Euclidean spaces \cite{tent}. Let us highlight some of its features: 
\begin{itemize}[leftmargin=*]
	\item The adjoint system of the discrete-time PMP on matrix Lie groups corresponding to the states \m{q} evolves on the dual \m{\liea^*} of the Lie algebra \m{\liea} despite the fact that the state dynamics \eqref{eq:sys} evolves on the Lie group \m{\lieg}. 

\item The adjoint system is linear in the adjoint variables \m{\left(\zeta^t, \xi^t \right)} because the maps \m{\liea^* \ni \zeta^t \mapsto \ad{\e^{-\mathcal{D}_{\zeta} \ham{\left(\optraj{t} \right)}}}^*\left( \zeta^t \right) \in \liea^*} and \m{T^*_q G \ni \mathcal{D}_{q} \ham{\left(\optraj{t} \right)} \mapsto \triv{\op{q}_t} \Bigl(\mathcal{D}_{q} \ham{\left(\optraj{t} \right)}\Bigr)} are linear for all \m{t.}

\item The assertions the ``Hamiltonian non-positive gradient condition'', the ``complementary slackness condition'', the ``non-positivity condition'', and the ``non-triviality condition'' are identical to the discrete-time PMP on Euclidean spaces. 
\end{itemize}
\begin{remark}
\begin{enumerate}[label=(\alph*), leftmargin=*]
\item Assumption \ref{ass:asm} is not the most general set of hypotheses for which we can derive a discrete-time PMP on matrix Lie groups. Continuous differentiability of  the functions \m{s_t, f_t, g_t, c_t,c_N} suffices for Theorem \ref{thm:DMP} to hold. 

\item Due to certain fundamental differences between discrete-time and continuous-time optimal control problems, the standard Hamiltonian maximization condition in continuous time \cite[Theorem MP on p.\ 14]{Sussmann_copmp} does not carry over to the discrete-time version. For a detailed discussion, see e.g., \cite[p.\ 199]{pshenichnyi1971necessary}. We do, however, get a weaker version contained in assertion \ref{it:hamnpg} of Theorem \ref{thm:DMP}. If the Hamiltonian is concave in control \m{u} over the set of feasible control actions \m{\cset} and \m{\cset} is compact for each \m{t}, in addition to \ref{asm:3}, the assertion can be strengthened to 
\begin{align*}
\ham{\left(t,\zeta^t,\xi^t, \op{q}_t,\op{x}_t, \op{u}_t\right)} = \underset{w \in \cset}{\max} \quad \ham{\left(t,\zeta^t,\xi^t, \op{q}_t,\op{x}_t, w\right)}.
\end{align*}
\end{enumerate}
\end{remark}
%%%%%%%%%%%%%%%%%%%%%%%%%%%%%%%%%%%%%%%%%%%%%%%%%
%
%%%%%%%%%%%%%%%%%%%%%%%%%%%%%%%%%%%%%%%%%%%%%%%%%
% 


