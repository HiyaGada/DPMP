\documentclass[12pt]{article}

%
     % \usepackage{/Users/ravibanavar/PBMacBook_26_05_09/TEX_myhelp/macros}        
       \usepackage{LGobservers_macros,defs} 					
       \usepackage{epsfig} 
         \usepackage{graphicx}
         \usepackage{graphics}
         \usepackage{latexsym} 
         \usepackage{amsmath}
         \usepackage{psfrag}
         \usepackage{amsmath}
         \usepackage{amssymb}
         \usepackage{multicol}
         \usepackage{cite}
         \usepackage{url}
         \usepackage{bm} % bold math
         \usepackage{color} % change text color  
         \usepackage{tcolorbox}
%
          \def\deff{\stackrel{\triangle}{=}} 
%
%       Page dimensions
%
     %\renewcommand{\baselinestretch}{1.2} 
         \topmargin=-0.75in%-0.5in 
      %\headheight=0pt 
	%\headsep=0pt 
	\textheight=9.5in 
	%\textwidth=6.5in 
	%\oddsidemargin=0pt 
	%\pagestyle{plain} 
	%\footskip=0.65in 
% 

\title{Discrete Filtering on Lie Groups using the Maximum Principle
\\ \small{Problem Formulation}}
\author{Ravi N. Banavar, Hiya Gada}
\date{August 2022}

\begin{document}

\maketitle
\thispagestyle{empty} 

We formulate a candidate filtering problem in discrete time on the rotation group 
$\SOthree$ using an optimal control framework 
and propose a "Maximum Principle" approach to solve the problem.

%\textcolor{red}{Are we also estimating the landmark positions? Are we considering landmarks on a sphere $\Stwo$? Yes, we are. Should we include the error in measurements in the objective?}


% Problem formulation includes:
% 1) Equations of system
% 2) Measurement
% 3) Error definition and dynamics
% 4) Estimate dependence on measurements
% 5) Objective function 
% 6) Something that brings in measurement estimation

\section{Equations of motion}
The governing kinematic equations of motion are
    \begin{align}
		R_{k+1} = R_k \exp (\hOmega{}_k T)
	\end{align}
 where $\hOmega{}_k$ is the angular velocity at the $k$th instant and $T$ is the sampling interval. 

\section{Measurement}
A measurement of the direction of an inertial object $\eta^i \in \Stwo$, where $i = 1, \dots, p$, is expressed in the body-frame of the robot as
 	\begin{align}
		y_k^i = R_k^T \eta^i
	\end{align}

\section{Observer Design}
The observed state at stage $k$ is denoted by $R_{e_k}$ and the observer
is assumed to have a structure
%
 	\begin{align}
		R_{e_{k+1}} &= R_{e_{k}} \exp (( \hOmega{} _k + \hM{}_k)T)\\
		R_{e_0} &= I_{3}
	\end{align}
 %%
where $\{ \hM_k : k = 0, \ldots, N-1\} $ is a sequence in $\sothree$ and further $\hM{}_l$ is a 
causal function of $ \{ y_k^i : k = 0, \ldots, l \text{ and } i = 1, \dots, p\} $.

\section{Error dynamics}
Define the error as,
\begin{align}
    E_k \deff R_{e_k} R_{k}^\top
\end{align}

The error dynamics are then,
\begin{align}
    E_{k+1} = \Ad_{R_{e_k}}   {\exp({\hM}_{k} T) }  E_k
\end{align}

\section{Dependence of $\hM_{k}$ on measurements}
Let there be a linear dependence on the measurements taken till the $k_{th}$ instant.
\begin{align}
    M_{k} = L_k(y_k - y_{e_k}), \;\; L_k \in \mathbb{R}^{3 \times 3p}
\end{align}
where $y_k$ and $y_{e_k}$ is a concatenation of all $y_k^i$,
\begin{align}
    y_k = \begin{bmatrix}
    y_k^1 \\
    y_k^2 \\
    \vdots \\
    y_k^p
    \end{bmatrix}, \;\; y_{e_k} = \begin{bmatrix}
    y_{e_k}^1 \\
    y_{e_k}^2 \\
    \vdots \\
    y_{e_k}^p
    \end{bmatrix}
\end{align}
The estimated measurement comes from the equation,
\begin{align}
    y_{e_k}^i = R_{e_k}^\top \eta^i
\end{align}

\section{Objective function}

The optimal observer problem would beas follows,
\begin{align*}
    \min_{M_k: k=0, 1, \ldots, N - 1} &\sum_{i=0}^{N} \trace{(E_i - I)}
	\nonumber \\
	\hbox{subject to } E_{k+1} &= \Ad_{R_{e_k}}   {\exp({\hM}_{k} T) }  E_k,
	\nonumber \\ 
	R_{e_{k+1}} &= R_{e_k} \exp (( \hOmega{} _k + \hM{}_k)T),
	\nonumber \\
	M_k & \in \mathbb{R}^{3\times3}, \;\; \forall \;\; k \in \{0, \cdots, N-1\}\\
	\hbox{\color{red}{How to }} & \hbox{\color{red}{define initial conditions?}}\\
	y_0, & y_1, \ldots, y_{N-1} \quad \hbox{given}
	\nonumber \\
	\hOmega{}_0, & \hOmega{}_1, \ldots, \hOmega{}_{N-1} \quad \hbox{given}
\end{align*}

\color{red}{We have a serious problem with boundary conditions. As this is an estimation problem we don't have knowledge of the true robot pose $R_k$. We cannot put a user defined intial estimate of $R_{e_k}$ and $E_k$ together as this will fix the true robot pose $R_k$ (which we have no knowledge of). We can only fix one of the variables. Can we solve this boundary value problem? How do we define final conditions?}

\color{purple}{Maybe we can define $R_{e_0}$ and $E_N$ (as we expect it to be 0 or close to 0 at the last time step)? Should we use "sweeping method"? }

% REFER
% multiple shooting methods, psuedo spectral methods, bryson and ho (Sweeping method)

\section{Discrete-time PMP}

Define the Hamiltonian,
\begin{align*}
    H^\nu(k, \zeta, R_{e}, E, M) &= \nu(\trace{(E - I)}) + \langle \zeta, \exp{}^{-1}() \rangle_{\mathfrak{g}} 
\end{align*}

\end{document}
